\part{Commutative Geometry with a View Toward Algebraic Geometry, Eisenbud}

\section{Preliminaries}
\subsection{Rings and Ideals}
\begin{definition}
    \textbf{Ring}: A ring is an abelian group $R$ with a multiplication operation $*: R\times R\to R$ as well as an identity element $1\in R$ such that:
    \[a(bc) = (ab)c \quad \forall a,b,c\in R\]
    \[a(b + c) = ab + ac \quad \forall a,b,c \in R\]
    \[(b + c)  = ba + ca \quad \forall a,b,c\in R\]
    \[1a = a1 = a \quad \forall a\in R\]
\end{definition}
A ring is commutative if the ring commutes with respect to multiplication, that is $ab = ba \quad \forall a,b\in R$.

\begin{definition}
    \textbf{A Unit in a Ring: }Let $R$ be a ring. An element $u\in R$ is a unit if it is invertible, that is there exists some $v\in R$ such that $vu = 1 \in R$. 
\end{definition}

\begin{proposition}
    \textbf{Uniqueness of the Multiplicative Inverse of an Element in a Ring: }Let $u\in R$ where $R$ is a ring. Then $us = ut = 1\implies s = t$, i.e. inverses are unique in $R$ and we can speak of 'the' inverse of $u$.
\end{proposition}

\textit{Proof} Consider the same set up. Then we have $su = 1 = ut$ \footnote{It seems that we have used commutativity of $R$ here but we have not. If $us = 1$ then $(su)s = s(us) = s \implies su = (su)s s^{-1} = s s^{-1} = 1$}. We then have
\begin{equation*}
\begin{split}
    s &= s1 \\
    &= s (ut) \\
    &= (su) t \\
    &= t
\end{split}
\end{equation*}
So we are done.

\begin{definition}
    \textbf{Field: }A field is a non-zero ring such that every non-zero element is invertible. 
\end{definition}

\begin{definition}
    \textbf{Zero Divisor of a Ring: }Let $R$ be a ring. A zero-divisor in $R$ is a non-zero element $u$ such that there is another non-zero element $s$ with $us = 0$
\end{definition}

Whilst this seems rather abstract, zero divisors crop up more frequently than you would imagine. For example if we consider the hours on a clock, with the multiplication operation between hours being the usual one (that is the hour 3 multiplied by the hour 5 is the hour 15, but on a clock this would be the hour 3), then we have zero-divisors for any integer $n$ such that there is some $k$ with $nk = 12m$ for some $m\in \{0,...,11\}$. For example $3,4$ are a pair of zero-divisors, $2,6$ is another example.

\begin{definition}
    \textbf{Ideal of a Ring: }Let $R$ be a ring. An ideal $I$ in $R$ is an additive subgroup such that if $r in R, s\in I$ then $rs \in I$. An ideal $I$ is said to be generated by the subset $\mathcal{S} \subseteq I$ if any element in $I$ can be expressed as a linear combination (over $R$) of elements in $S$. More specifically,
    \[\exists \theta_1,...,\theta_n \in R, s_1,...,s_n \in \mathcal{S}: r = \sum_{i = 1}^n \theta_i s_i\]
\end{definition}

Some important notes here. A ring is \textbf{principal} if it is generated by a single element, in which case we write $I = (s)$. An ideal $I \subset R$ is prime if for any $f,g \in R$, if we have $fg\in I$ then either $f$ or $g$ is in $I$. A ring $R$ is a domain if $(0)$ is a prime ideal.\footnote{This seems like a weird definition at first, but it is equivalent to not having any zero-divisors. If $fg\in (0)$ then $(0)$ prime would men either $f= 0$ or $g = 0$, i.e. no zero-divisors} A maximal ideal of $R$ is a proper ideal $\mathfrak{m}$ that is not contained in any other ideal. Moreover, if $\mathfrak{m}$ is a maximal ideal, then $R/\mathfrak{m}$ is a field. 

\begin{proposition}
    Let $R$ be a ring and $\m$ a maximal ideal of $R$. Then $R/\m$ is a field. 
\end{proposition}

\begin{definition}
    \textbf{Commutative Algebra over a Ring: }Let $R$ be an abelian ring. A commutative algebra over $R$ is a commutative ring $S$ with a ring homomorphism $\alpha: R\to S$.
\end{definition}

\begin{proposition}
    Any ring is an algebra over the ring over integers $\Z$. 
\end{proposition}

\begin{definition}
    \textbf{Subalgebras: }Let $S$ be an algebra over a commutative ring $R$. A subring $S'$ is a commutative $R$-subalgebra of $S$ if $\text{Im}(\alpha) = \alpha(R) \subset S'$
\end{definition}

A homomorphism of $R$-algebras $\phi : S\to T$ is a homomorphism of rings such that $\phi(rs) = r\phi(s) \quad \forall r\in R, s\in S$. 

\subsubsection{Unique Factorisation}
Let $R$ be a ring. An element $r\in R$ is irreducible if it is not a unit and $r = st$ implies that one of $s,t$ is a unit in $R$. A ring $R$ is a \ufd  if any factorisation is unique up to scaling by units in $R$. 

\subsubsection{Modules}
\begin{definition}
    \textbf{Modules over Rings: }Let $R$ be a ring. An $R$-module $M$ is an abelian group together with an action with $R$, i.e. a map $\R\times M\to M$ expressed as $(r,m)\to rm$ satisfying $\forall r,s\in R, mn\in M$:
    \begin{itemize}
        \item $r(sm) = (rs)m$ 
        \item $r(m + n) = rm + rn$ 
        \item $(r + s)m = rm + sm$
        \item $1m = m$
    \end{itemize}
\end{definition}
The most interesting $R$-modules are those that take the form of ideals $I$ and their corresponding factor rings $R/I$. If $M$ is an $R$-module then the annihilator of $M$ is 
\[\ann_R(M) := \{r\in R: rM = 0\}\]
An example of which is $\ann_R(R/I) = I$ for any ideal $I\subset R$. We can generalise this notion of quotients. Let $I,J$ be ideals of $R$, we write $(I:J) = \{f \in R: fJ\subset I\}$. Generalising further we get the notion of submodules. Let $M, N$ be submodules of an $R$-module $P$, we write $(M:N) = \{f\in R: fM \subset N\}$. 

If $M,N$ are $R$-modules then the direct sum $M \oplus N$ is the module $M\oplus N = \{(m,n): m\in M, n\in N\}$. There are the natural inclusion maps $M \xhookrightarrow{} M \oplus N, m \to (m,0)$ and projection maps $\pi: M\oplus N \to M, (m,n)\to m$. If we have existence of maps $\alpha: M\to P, \sigma: P\to M, \sigma\circ\alpha = {id}_P, \alpha\circ\sigma = {id}_M $ then we say $M$ is a direct summand of $P$. In this case we actually have a nice formula,
\[P \simeq M \oplus \ker \sigma \]
The simplest form of $R$-modules are just direct sums of the original ring. Modules of this form are called free modules (over $R$). A small digression is made here. The direct product of $R$-modules $M_i$, $\prod_i M_i$ is the set of tuples $(m_i)$ whereas the direct sum is $\oplus_i M_i \subset \prod_i M_i$ where an element $\tilde{m} \in \oplus_i M_i$ is an n-tuple with the additional constraint that all but finitely many are equal to $0$. 

A free $R$-module is a module that is isomorphic to a direct sum of copies of $R$. If $M$ is a finitely generated free $R$-module then $M \cong R^n$ for some $n\in \N$. IF $A,B,C$ are $R$-modules and $\alpha : A\to B, \beta:B\to C$ are homomorphisms, then a sequence 
\[A\stackrel{\alpha}{\to} B \stackrel{\beta}{\to} C\]
is exact if $Im(\alpha) = \ker(\beta)$. In general a sequence
\[0\to A_1 \to A_2 \to ... \to A_n\]
is exact if $\ker(\phi_i: A_i\to A_{i+1}) = Im(\phi_{i-1}: A_{i - 1}\to A_i)$
A short exact sequence is an exact sequence of the form 
\[0\to A\stackrel{\alpha}{\to} B \stackrel{\beta}{\to} C\to 0\]
Some nice examples follow. If $M_1, M_2$ are submodules of $M$ then $M_1 + M_2 \subset M$ is also a submodule. We get the short exact sequence
\[0\to M_1\cap M_2 \stackrel{\iota}{\to} M_1\oplus M_2 \stackrel{(m_1,m_2)\to m_1 - m_2}{\to} M_1 + M_2 \to 0\]
\section{Basic Constructions}
\subsection{Localisation}
A local ring is a ring with a single unique maximal ideal. The technique of localisation reduces many problems in commutative algebra to problems on commutative rings. The idea of localisation is as follows. Given a point $p$ in an algebraic set $X \subset \Aff^n_k$, we want to investigate what $X$ looks like near $p$, that is we want to investigate arbitrarily small open neighbourhoods of $p$ in the Zariski topology. The Zariski open neighbourhoods of $p$ are sets of the form $X\backslash Y$ for $p\not in Y \subset X$. 

\subsubsection{Fractions}
\begin{definition}
    \textbf{Localisation of an $R$-Module via a Multiplicatively Closed Subset $U$: }Let $R$ be a ring, $M$ an $R$-module and $U\subset R$ a multiplicatively closed subset. The localisation of $M$ at $U$, $M[U^{-1}]$ is the set of equivalent classes of pairs $(m,u) \sim (m', u')$ where $m,m' \in M, u,u' \in U$ are related if there is some $v\in U$ such that $v(mu' - m'u) = 0$. 
\end{definition}

\begin{proposition}
Let $U$ be  multiplicatively closed set of $R$ and let $M$ be an $R$-module. An element $m\in M$ goes to $0$ in $M[U^{-1}]$ under the map $\pi: M\to M[U^{-1}], m\to m/1$ if $m$ is annihilated by an element $u\in U$. In particular, if $M$ is finitely generted then $M[U^{-1}] = 0$ iff $M$ is annihilated by an element of $U$.
\end{proposition}
\textit{Proof} Let $Ann_R(m) = \{r\in R: rm = 0\}$ be the \textit{annihilator of $M$ in $R$}. Then $m\to m/1 \in M[U^{-1}]$ maps to $0$ if it is equivalent to $0$ under the relation $~$, that is to say $m/1\sim 0 \iff \exists u\in U : u(m - 0) = um = 0$. That is the annihilator of $m$ is some subset of $U$.

The first example of localisation is \textbf{quotient field of an integral domain}. Let $R$ be an integral domain, and take the localisation of $R$ with respect to $U = R\backslash \{0\}$. This localisation, $R[U^{-1}] =: K(R)$ is the \textbf{total quotient ring of $R$}. 

If $P$ is a prime ideal of $R$ and $U = R \backslash P$, then we have another localisation $R[U^{-1}]$. Let $R$ be the coordinate ring of a variety $X$ then the local ring of $X$ at a point $x\in X$ is then the local ring found via inverting any elements that don't vanish at $x$. Recall that a point $x\in X$ corresponds to a prime ideal $\m_x$ of functions that vanish at $x$. Then the local ring (which I learnt denoted as $\mathcal{O}_{x,X}$) is $R[(R\backslash P)^{-1}]$. 

We can compute some examples. Let $X = \V(x^2 + y^2 - 1)\subset \Aff^2$. The local ring at $(1,0)$ is then:

\begin{equation*}
    \begin{split}
        \Lr_{(1,0), X} &:= (K[x,y]/(x^2 + y^2 - 1))_{(x- 1, y)} 
    \end{split}
\end{equation*}
This seems rather abstract but we can directly compute the unique maximal ideal. The maximal ideal $m_{(1,0)} = (x - 1, y)$ has $m_{(1,0)}^2 = ((x-1)^2, (x-1)y, y^2)$. Under the relation generated in the coordinate ring we know that $x^2 + y^2 = 1$:
\begin{equation*}
    \begin{split}
        x^2 + y^2 - 1 &= (x-1)(x+1) + y^2 \\
        \implies x - 1 &= y^2/(x+1)
    \end{split}
\end{equation*}
i.e. that $x-1\in m_{(1,0)}^2$, so $m_(1,0)^2 = (y)$ and $\Lr_{(1,0), X}$ is completely generated by $y$. There are a couple of things to note here. In this case, when $t \in m_p \backslash m_p^2$ is a generator we say $t$ is a uniformiser for the maximal local ring. Also, have have just shown that the local ring is $1$ dimensional as viewed as a vector space over $K[X]/\m_p$. This is an algebraic criterion for non-singularity, that is the point $(1,0) \in X$ is a non-singular point. 



\subsubsection{Hom and Tensor}
\subsubsection{The Construction of Primes}
\subsubsection{Rings and Modules of Finite Length}
\subsubsection{Products of Dom}

\subsection{Associated Primes and Primary Decomposition}
\subsubsection{Associated Primes}
\subsubsection{Prime Avoidance}
\subsubsection{Primary Decomposition}
\subsubsection{Primary Decomposition and Factorality}
\subsubsection{Primary Decomposition in the Graded Case}
\subsubsection{Extracting Information form Primary Decomposition}
\subsubsection{Why Primary Decomposition is not Unique}
\subsubsection{Geometric Interpretation of Primary Decomposition}
\subsubsection{Symbolic Powers and Functions Vanishing to High Order}

\subsection{Integral Dependence and the Nullstellensatz}
\subsubsection{The Cayley-Hamilton Theorem and Nakayama's Lemma}
\subsubsection{Normal Domains and the Normalisation Process}
\subsubsection{Normalisation in the Analytic Case}
\subsubsection{Primes in an Integral Extension}
\subsubsection{The Nullstellensatz}

\subsection{Filtrations and the Artin-Rees Lemma}
\subsubsection{Associated Graded Rings and Modules}
\subsubsection{The Blowup Algebra}
\subsubsection{The Krull Intersection Theorem}
\subsubsection{The Tangent Cone}

\subsection{Flat Families}
\subsubsection{Elementary Examples}
\subsubsection{Introduction to Tor}
\subsubsection{Criteria for Flatness}
\subsubsection{The Local Criterion for Flatness}
\subsubsection{The Rees Algebra}

\subsection{Completions and Hensel's Lemma}
\subsubsection{Examples and Definitions}
\subsubsection{The Utility of Completions}
\subsubsection{Lifting Idempotents}
\subsubsection{Cohen Structure Theory and Coefficient Fields}
\subsubsection{Basic Properties of Completion}
\subsubsection{Maps from Power Series Rings}

\section{Dimension Theory}

\subsection{Introduction to Dimension Theory}
\subsection{Fundamental Definitions of Dimension Theory}
\subsection{The Principal Ideal Theorem and Systems of Parameters}
\subsection{Dimension and Codimension One}
\subsection{Dimension and Hilbert-Samuel Polynomials}
\subsection{The Dimension of Affine Rings}
\subsection{Elimination Theory, Generic Freeness and the Dimension of Fibres}
\subsection{Gröbner Bases}
\subsection{Modules of Differentials}

\section{Homological Methods}
\subsection{Regular Sequences and the Koszul Complex}
\subsection{Depth, Codimension and Cohen-Macaulay Rings}
\subsection{Homological Theory of Regular Local Rings}
\subsection{Free Resolutions and Fitting Invariants}
\subsection{Duality, Canonical Modules and Gorenstein Rings}

