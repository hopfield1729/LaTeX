\part{Sheaf Theory, Bredon}

\section{Sheaves and Presheaves}
\subsection{Definitions}
\begin{definition}
    \textbf{Presheaf: }A presheaf of abelian groups on a topological space $X$ is a functor
    \[A: \mathfrak{Opn}_X^{op} \to \mathfrak{Ab}\]
    i.e. an assignment to each open set $U \in Ob (\mathfrak{Opn}_X^{op})$ an abelian group $A(U) \in Ob(\mathfrak{Ab})$, and on inclusion functions $V \xhookrightarrow{\iota^V_U} U$ restriction functions $r^U_V: A(U) \to A(V)$ such that:
    \begin{itemize}
        \item $r^U_U = id_U$,
        \item $U \subset V \subset W \implies r^W_V r^V_U = r^W_U$
    \end{itemize}
\end{definition}

One can switch out the target category with other algebraic categories to obtain the definitions of presheaves over $R$-modules, algebras, etc. 

\begin{definition}
    \textbf{Germ: }Let $x \in X$ be a point and $A$ a presheaf over $X$. The set $\mathfrak{M}$ of elements $s \in A(U)$, $x \in U \in Ob \mathfrak{Opn}_X^{op}$. The germs of the presheaf $A$ at $x$ is thus the quotient   
    \[\mathfrak{G}_{A,x}:= \mathfrak{M}/\sim \]
    where $s,t \in \mathfrak{M}$ are equal if there is $W \subset \text{dom}(s) \cap \text{dom}(t)$ such that $s|_W = t|W$. The equivalence class $[s]$ of germs of $A(U)$ is called the germ of $s$ at $x \in U$.
\end{definition}

The set of germs of $A$ at $x$, $\mathcal{A}_x$ is then the direct limit
\[\mathcal{A}_s = \lim_{\to} A(U)\]
From here we can define a topology on the disjoint union $\sqcup_{x\in X} \mathcal{A}_x$ as follows. Fix an element $s\in A(U)$, for each $x \in U$ we have the germ $s_x$ of $s$ at $x$. For fixed $s$ the set of all germs $s_x \in \mathcal{A}_x$ is defined to be an open set. With this definition we get the sheaf generated by the presheaf $A$L
\[\mathcal{A} = \mathfrak{Sheaf}(A) = \mathfrak{Sheaf}(U \to A(U))\]

\begin{definition}
    \textbf{Sheaf: }A sheaf on $X$ is a pair $(\mathcal{A}, \pi)$ satisfying:
    \begin{itemize}
        \item $\mathcal{A}$ is a (typically non-Hasuforff) topological space,
        \item $\pi: \mathcal{A}\to X$ os a local homeomorphism
        \item Each $\mathcal{A}_x = \pi^{-1}(x)$ for $x\in X$ is an abelian group, and is called the stalk of $\mathcal{A}$ at $x$
        \item The group operations are continuous
    \end{itemize}
\end{definition}
An example of a sheaf is $\Omega^0$ of germs of $\mathcal{C}^\infty(\R)$ functions on a differentiable manifold $M^n$, this is a sheaf of unital rings, and $\Omega^p(M^n)$ of germs of differential $p$-forms on $M^n$ which is also a $\Omega^0$-module.

If $\mathcal{A}$ is a sheaf on $X$ with projection $\pi: \mathcal{A}\to X$ and if $Y\subset X$ then the restriction $\mathcal{A}|Y$ of $\mathcal{A}$ is defined:
\[\mathcal{A}|Y = \pi^{-1}(Y)\]
If $\mathcal{A}$ is a sheaf on $X$, $Y \subset X$ then a section of $\mathcal{A}$ over $Y$ is a map $s: Y\to\mathcal{A}$ such that $\pi \circ s$ is the identity on $Y$. Note that every point $x\in Y$ admits a section $s$ over some neighborhood $U$ of $x$. We can then access the $0$ section over any open set $U \subset X$ and so we can endow $\mathcal{A}(Y)$ with the structure of an abelian group. Thus the presheaf of sections of $A$ is defined:
\[\Gamma(\mathcal{A}) = \mathcal{A}(X)\]

\subsection{Homomorphisms, Subsheaves and Quotient Sheaves}
\subsection{Direct and Inverse Images}
\subsection{Cohomomorphisms}
\subsection{Algebraic Constructions}
\subsection{Supports}
\subsection{Classical Cohomology Theories}

\section{Sheaf Cohmology}
\subsection{Differential Sheaves and Resolutions}
\subsection{The Canonical Resolution and Sheaf Cohomology}
\subsection{Injective Sheaves}
\subsection{Acyclic Sheaves}
\subsection{Flabby Sheaves}
\subsection{Connected Sequences of Functors}
\subsection{Axioms for Cohmomology and the Cup Product}
\subsection{Maps of Spaces}
\subsection{$\Phi$-Soft and $\Phi$-Fine Sheaves}
\subsection{Subspaces}
\subsection{The Vietoris Maping Theorem and Homotopy Invariance}
\subsection{Relative Cohomology}
\subsection{Mayer-Vietoris Theorems}
\subsection{Continuity}
\subsection{The Künneth and Universal Coefficient Theorems}
\subsection{Dimension}
\subsection{Local Connectivity}
\subsection{Change pf Supports and Local Cohomology Groups}
\subsection{The Transfer Homomorphism and the Smith Sequences}
\subsection{Steenrod's Cyclic Reduced Powers}
\subsection{The Steenrod Operations}

\section{Comparison with Other Cohomology Theories}
\subsection{Singular Cohomology}
\subsection{Alexander-Spanier Cohomology}
\subsection{de Rham Cohmomology}
\subsection{Cech Cohomology}

\section{Applications of Spectral Sequences}
\subsection{The Spectral Sequence of a Differential Sheaf}
\subsection{The Fundamental Theorems of Sheaves}
\subsection{Direct Image Relative to a Support Family}
\subsection{The Leray Sheaf}
\subsection{Extension of a Support Family by a Family on the Base Space}
\subsection{The Leray Spectral Sequence of a Map}
\subsection{Fiber Bundles}
\subsection{Dimension}
\subsection{The Spectral Sequences of Borel and Cartan}
\subsection{Characteristic Classes}
\subsection{The Spectral Sequence of a Filtered Differential Sheaf}
\subsection{The Fary Spectral Sequence}
\subsection{Sphere Bundles with Singularities}
\subsection{The Oliver Transfer and the Conner Conjecture}

\section{Borel-Moore Homology}
\subsection{Cosheaves}
\subsection{The Dual of a Differential Cosheaf}
\subsection{Homology Theory}
\subsection{Maps of Spaces}
\subsection{Subspaces and Relative Homology}
\subsection{The Viertoris Theorem, Homotopy and Covering Spaces}
\subsection{The Homology Sheaf of a Map}
\subsection{The Basic Spectral Sequence}
\subsection{Poincaré Duality}
\subsection{The Cap Product}
\subsection{Intersection Theory}
\subsection{Uniqueness Theorems}
\subsection{Uniqueness Theorems for Maps and Relative Homology}
\subsection{The Künneth Formula}
\subsection{Change of Rings}
\subsection{Generalised Manifolds}
\subsection{Locally Homogenous Spaces}
\subsection{Homological Fibrations and $p$-adic Transformation Groups}
\subsection{The Transfer Homomorphism on Homology}
\subsection{Smith Theory in Homology}

\section{Cosheaves and Cech Homology}
\subsection{Theory of Cosheaves}
\subsection{Local Triviality}
\subsection{Local Isomorphisms}
\subsection{Chech Homology}
\subsection{The Reflector}
\subsection{Spectral Sequences}
\subsection{Coresolutions}
\subsection{Relative Cech Homology}
\subsection{Locally Paracompact Spaces}
\subsection{Borel-Moore Homology}
\subsection{Modified Barel-Moore Homology}
\subsection{Singular Homology}
\subsection{Acyclic Coverings}
\subsection{Applications to Maps}
